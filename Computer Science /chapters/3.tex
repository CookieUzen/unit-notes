\documentclass[../notes.tex]{subfiles}

\begin{document}
	
\section*{Vocabulary}
\begin{itemize}
	\item Bus Topology: Computer network in which a ``bus'' connects all the devices together with a cable
	\item Cable: a wire that is used for networking
	\item Check Digit: Extra digit added to check data integrity after input, transmission, storage, or processing
	\item Data Integrity: Accuracy of data after input, transmission, storage, or processing
	\item Data Packet: Portion of a message that is transmitted through a network. Contains data such as check digits and destination address.
	\item Gateway: Link that resides between computer networks that convert data so that it is understood by the receiving network.
	\item Handshaking: exchange of predetermined signals to signify that a connection has been established between two systems.
	\item Hub: Network connection point for devices. Data arriving at a hub is copied over and sent to all the devices on the network.
	\item ISDN (Integrated Services Digital Network): An international communication standard that allows for the transmission of media over digital telephone lines.
	\item LAN (Local Area Network): A computer network where the computers are connected locally. 
	\item Microwave Transmission: Wireless electronic communication
	\item Modem (Modulator/Demodulator): A device that converts electrical device to audio signals that travels on telephone lines.
	\item Network: Computer systems that are interconnected. 
	\item Packet: Group of bits that form some sort of data. 
	\item Packet Switching: method of network communication that transmit packets through a network.
	\item Networking: Using a network. 
	\item Parity bit: Error detecting bit that is appended on data in order to check for data integrity
	\item Protocol: International rules that ensure the transfer of data between system.
	\item TCP/IP (Transmission Control Protocol/Internet Protocol): Communication protocol used for connection on the Internet.
	\item WAN (Wide Area Networks): A connected network of LANs.
\end{itemize}

\section{Network Fundamentals}

\subsection{Different Types of Computer Networks}
In a computer network, there is often a \textit{Server} and a \textit{Client}. 
A \textit{Server} provides services to the \textit{client}.

In a network, there is three important device.
A \texttt{hub} is a connection point for devices on a network. 
Network devices are connected to the hub through a cable.
When a client request some data from a server, the hub copies the data and send it to all devices connected to it. 
This is sometimes problematic, as it causes extra traffic in the network.

Similar to a hub, a \texttt{switch} is a connection point for devices on a network.
However, a switch is able to identify the device on each port, and transmit data accordingly to the receiver.
Nowadays, hubs are obsolete.

A \texttt{router} is device that join multiple networks.
One potential job of a router is joining the LAN with the Internet.
A simple switch cannot accomplish this job.
However, nowadays, routers and switch are mostly integrated into one device, allowing the creation of wireless networks connected to wired ones.

Below is a list of the common network types:
\begin{itemize}
	\item LAN
	\item WLAN (Wireless LAN)
	\item VLAN (Virtual LAN)
	\item WAN
	\item SAN (Storage Area Network)
	\item Internet
	\item Extranet
	\item VPN (Virtual Private Network)
	\item PAN (Personal Area Network)
	\item Peer to Peer (P2P)
\end{itemize}

In fact, the Internet is a really big WAN that is connected to thousands of LAN around the world.

\subsubsection{Local Area Network}
In a LAN, peripheral devices and data can be shared.
There is a high data-transfer rate between computers on the network.

\subsubsection{WLAN}
A WLAN is just a wireless LAN.
WLANs generally are not as fast as an wired network, but allows the wireless devices to move freely.
However, the use of WiFi also reduce the safety of the data transfered. 
To work around this, secure protocals (WEP, WPA, WPA2) should be used.

\subsubsection{VLAN}
VLAN stands for Virtual Local Area Network.
A VLAN artificially divides a network into separate compartments.
Logically separated computer networks are similar to physically separate networks.
Unless separate rules have been set, the computer cannot see across a VLAN.
VLANs allow for easy management of the resources on a network, and improves security.

\subsubsection{WAN}
A WAN connects many LANs across a wide geographic area.
For example, a company may have a WAN connecting its many headquarters around the world.

\subsubsection{SAN}
A SAN network is a network specifically designed for the sharing of large storage devices.

\subsubsection{Intranet and Extranet}
An intranet is a collection of private computer networks within an organization that is not connected to the Internet.
Intranets are secure and fast, but hard to access outside of the LAN.

Similarly, an Extranet is a collection of private computer network, but opened to the Internet.
This is less secure, but allows for more flexibility.

\subsubsection{Internet}
The Internet is just a very large WAN connecting billions of LANs around the world together.

\subsubsection{IoT}
Internet of Things are small devices that can connect to the Internet. 
These small devices may send and receive data.

\subsubsection{VPN}
A Virtual Private Network allows for an encrypted tunnel to be established between two computers.
This allows for remote workers to access the resources on a company Intranet while ensuring safety of data.

\subsubsection{PAN}
A Personal Area Network is composed of devices that are centered around an individual's workspace.
These devices have very short ranges.

\subsubsection{P2P}
A Peer to Peer network do not uses the client/server network, but instead a distributed network architecture where all computer systems are decentralized.
The same device can be both a server and a client.

\subsection{Importance of Standards}
There are two main organization setting standards for computer communication.
Institute of Electrical and Electronics Engineers (IEEE) and Internet Engineering Task Force (IETF).
The standards they create help create compatibility between hardware and software, computers of different brands and origins.
Without standards, communication between computers are nearly impossible.

\subsection{Networks, Communication, and Layers}
In order to simplify the many layers of inter computer communication, a network is split into a different layers, where each layer represents a module.
The \texttt{OSI Model} is created for this reason.
The OSI Model stands for Open System Interconnection Model.

The OSI Model is split into 7 parts. 
\begin{enumerate}
	\item Application: Various services used by the end user
	\item Presentation: Provides data format, compression, encryption information.
	\item Session: Manages a connection between two users
	\item Transport: Definition of data segments, assignment of numbers, data transfer, re assemblage of data 
	\item Networking: Handling packet switching
	\item Data Link: Error handling for transitions, transmission rate, flow control
	\item Physical: Transmission of 1 and 0 through cables and microwave. Definition of media and specification.
\end{enumerate}

The \texttt{TCP/IP} protocal describes functions that take place in each layer. 
\begin{enumerate}
	\item Application: Perform services for software.
	\item Transport: The transformation of data.
	\item Internet: Packet Switching
	\item Network Access: Media and devices
\end{enumerate}

\subsection{Technology required for a VPN}
Hardware requirements
\begin{itemize}
	\item Internet Access
	\item VPN Software
	\item VPN Routers
	\item VPN Appliances
	\item VPN Concentrators (handles large number of incoming VPN tunnels)
	\item VPN Servers
\end{itemize}

For \textit{Secure} VPN (encrypted, tunneled VPN)
\begin{enumerate}
	\item IPSec (Internet Protocal, Security Protocal)
		\begin{itemize}
			\item AES (Encryption)
		\end{itemize}
	\item SSL (Secure Socket Layer) or TLS (Transport Layer Security)
\end{enumerate}

For \textit{Trusted} VPN (Secure on provider side)
\begin{enumerate}
	\item ATM (Asyncronous Transfer Mode) circuits
	\item Frame delay circuits
	\item Transport of layer 2 frames over MultiProtocol Label Switching
\end{enumerate}

\texttt{Hybrid} VPNs are a mix of Trusted and Secure VPN.

\subsection{Common VPN Types}
\begin{itemize}
	\item Site to site VPN (Connecting network or facilites together)
	\item Remote access VPN (remote into an intranet)
\end{itemize}

\subsection{Uses of a VPN}

VPN Benefits
\begin{itemize}
	\item Easy Communication
	\item Secure connection
	\item Decrease operational cost when compared to WAN
	\item Extended connection
	\item Allows for remote work
	\item Allows global networking
	\item Simplify network topology
\end{itemize}

\section{Data Transmission}
\subsection{Necessity of Protocols}
Requirements for communication to take place:
\begin{itemize}
	\item Presence of an identified sender
	\item Presence of an identified receiver
	\item Presence of an agreed-upon method of communicating
	\item Presence of a common language
	\item Presence of a common grammar
	\item Presence of an agreed-upon speed and timing of delivery
	\item Presence of confirmation or acknowledgment requirements
\end{itemize}

Computer Networks Protocols also provides:
\begin{itemize}
	\item Rules about the message format
	\item Rules about the way intermediary devices should facilitate communication
	\item Rules about initiation and termination of a communication session
	\item Rules about the type of error checking to be used
	\item Rules about an error detection and correction mechanism
	\item Rules about recovery and resending data
\end{itemize}

Computer Network Protocol guarantee:
\begin{itemize}
	\item Data Integrity
	\item Source Integrity
	\item Flow Control
	\item Congestion Management
	\item Deadlock Prevention
	\item Error Checking
	\item Error Correction
\end{itemize}

\subsection{Speed of Data Transmission across a network}
We measure the ability of a medium to transfer data as bandwidth.
Bandwidth is measured in $Mbps$, or Mega-bits per second.
This is different from $MBps$, which means Mega-Byte per second.

The theoretical speed of data is called bandwidth, the actual transfer rate is called throughput.
While the slowest part of a network is called the bottleneck.
Goodput measures the transfer rate of usable data.

There are several factors that affect the speed of data transmission in a network:
\begin{itemize}
	\item Bandwidth
	\item Data transfer rate of storage devices
	\item Interference
	\item Malicious software
	\item Number of connected deviecs
	\item Packet loss
	\item Bottleneck
	\item Client/Server specification
\end{itemize}

\subsection{Compression of Data}
In order to reduce bandwidth, compression is sometimes used to reduce the file size.
There is two types of compression, Lossy and Lossless.
Text have to be compressed losslessy, but some medias can be compressed with lossy compression.

\subsection{Characteristics of different transmission media}
There are three most common wires.
Fiber Optic cable, coaxial cables, and Unshielded Twisted Pair Cable (UTP).
There are many wireless communication methods.
\begin{itemize}
	\item Microwave Radio
	\item Satellites
	\item Infrared
	\item RFID
	\item Bluetooth
	\item Free Space Optics
\end{itemize}

In general, fiber optics is the fastest, most secure, and most reliable.
Wireless is not as secure, and not as fast or reliable.
UTP lies between the two.

\subsection{Packet Switching}
Packet switching is the communication method in which data packets are transfered through a network.
Packets are sent individually, and may take different paths to the same destination.
There is two method types of packet switching.

Datagram packet switching, where each packet is sent with an address. 
The route can be different per packet.

Virtual Circuit packet switching, where the route and the destination is predetermined before hand.

\subsection{Network Topology}
There are eight basic topologies:
\begin{itemize}
	\item Point to point
	\item bus
	\item star
	\item ring
	\item mesh
	\item tree
	\item fully connected
	\item hybrid
\end{itemize}

\section{Wireless Networking}

\subsection{Advantages and Disadvantages}
Advantages: 
\begin{itemize}
	\item Ease to setup
	\item Cheap
	\item Wireless/Convenient
	\item Easier to plan
\end{itemize}

Disadvantage:
\begin{itemize}
	\item Slow
	\item High error rates
	\item Weather dependent
	\item Weakest security
	\item Unreliable
\end{itemize}

\subsection{Hardware and Software Components}
Hardware needed
\begin{itemize}
	\item Modem 
	\item Wireless Router/Access Point
	\item Wireless Network Adapter (NIC)
	\item Device
	\item Wireless antennas
	\item Wireless repeater
	\item Ethernet repeater
	\item Ethernet over power line
\end{itemize}

Software needed
\begin{itemize}
	\item DHCP (IP Address assigning)
	\item Software Firewall
	\item Name/SSID (Service Set IDentification)
	\item NIC Drivers (Network Interface Card)
	\item Operating System
	\item Security Software
	\item WAP (Wireless Application Protocol)
	\item Web Browser
\end{itemize}

\end{document}
