\documentclass[../notes.tex]{subfiles}

\begin{document}
\section{Computer Architecture}
The Central Processing Unit (CPU) is an integral part of a computer.
The CPU follows the \texttt{input, process, output model}, and is responsible for doing most of the calculations our computers need to do in order to run.
The CPU is the brain of a computer.

A CPU contains:
\begin{itemize}
	\item Control Unit
	\item Arithmetic Logic Unit
	\item Memory Address Register
	\item Memory Data Register
\end{itemize}

CU is responsible for operation of the CPU.
It retrieves instructions from memory, then execute.

ALU performs math/logic/IO operations.
The CU provides ALU data and instructions.

MAR is responsible for the memory address of data needed, while MDR is responsible for retrieving the data from the memory address.

\section{Primary Memory}
\begin{itemize}
	\item 1 bit = one or zero
	\item 8 bits = 1 byte
	\item 64 bit system = 64 bits in each Memory Bus data transfer
	\item MB = Mega Byte, Mb = Mega bit
\end{itemize}

Memory is separated into \texttt{Random Access Memory} and \texttt{Read Only Memory}.

\subsection{RAM}
RAM stores program instructions and relevant data.
Data in RAM is stored in unique places: Memory Addresses.

Note, RAM allows for frequent read-write, and is volatile.
If power is lost, all information is lost.

RAM is separated into two different types: 
\begin{itemize}
	\item Dynamic RAM
	\item Static RAM
\end{itemize}

\subsubsection{DRAM}
DRAM is slower than SRAM, but cheaper.
This is the RAM sticks we use.

\subsubsection{SRAM}
SRAM is faster but more expensive (and smaller). 
SRAM is placed between the CPU and RAM.
It is also called a cache.

Cache holds useful information CPUs frequently uses.
This is so that the CPU does not have to use the slower DRAM when fetching instructions.
The CPU always read and write through the cache memory.

There are 3 levels of cache memory.
L1 is on the processor itself.
L2 lies between primary memory and the CPU.
Newer L3 cache is fulfilling similar roles to a L2 cache.

\subsection{ROM}
ROM holds instructions.
But ROM is read only, and stores permanent information.
Such as instructions for boot or the BIOS (Basic Input Output System).

Often, ROM stores embedded data that is non-volatile, and is used to store permanent programs.
ROM is also commonly smaller than RAM.

\section{Registers}
Registers are a small storage location that holds data.
We use the word \texttt{word} to indicate the register size (in bytes).

Both MAR and MDR are registers.
MAR stores memory address, MDR stores data (which is then saved to RAM).
A Memory (Address) Bus helps communicate between the MAR and the primary memory.
A Data Bus helps communicate between the MDR and the ALU.

\section{Machine Instruction Cycle}
This is the steps in which instructions are carried out in machine code.
\begin{enumerate}
	\item Fetch instructions from Primary Memory to Control Unit
	\item Decode instructions in Control Unit (loads additional data)
	\item Execute instructions
	\item Stores results and check for next instruction
\end{enumerate}

\section{Secondary Memory}
Slower memory than is non-volatile.
Used to store data permanently from RAM.

During startup, RAM is empty. 
And thus Secondary Memory is needed to load data onto RAM.

When primary memory is not enough for one application, Virtual Memory enables program to store their data on Secondary Memory.
Virtual Memory moves the memory of unused application to secondary memory, freeing up space for newly opened application.
The memory is loaded into primary memory back on when the unused application is accessed again.

Some examples of secondary memory are:
\begin{itemize}
	\item Hard Drive
	\item DVD/CD
	\item USB
	\item SD Card
	\item Floppy Disk
	\item Magnetic Tape
\end{itemize}

\section{Operating and application Systems}
An \texttt{Operating System} is a set of software that controls the computer's hardware resources and provides services for computer programs.
It interactive between the user and the hardware of the computer system.

The main services of an OS are:
\begin{itemize}
	\item Peripheral Communication
	\item Memory Management
	\item Resource Monitoring and Multitasking
	\item Networking
	\item Disk Access and Data Management
	\item Security
\end{itemize}

\section{Software Application}
Computer system uses software application to complete some task.
The main types of Software Application includes:

\begin{itemize}
	\item Word Processors
	\item Spreadsheets
	\item Database Management Systems
	\item Web Browsers
	\item Email
	\item Computer Aided Design
	\item Graphic Processing Software
\end{itemize}

\subsection{Word Processor}
A software used for production of any sort of document.
Used for composing, editing, formatting, and printing of a document.

\subsection{Spreadsheets}
Spreadsheets are programs used for the analysis of data.
Data is represented on cells, organized in rows and columns.
Spreadsheets are used for operation of arithmetic and mathematical functions.

\subsection{Database Management System}
A DBMS is an application application that manages a database.
Common features of a DBMS includes:
\begin{itemize}
	\item Create Queries
	\item Update Stores
	\item Modifies Queries
	\item Extract Information
\end{itemize}

This program provides an interface between a database between an user.
A database can often be represented as table, with rows and columns of data.

\subsection{Web Browser}
A web browser is a software that is used to access, retrieve, and present content on the Internet.

\subsection{Email}
Electronic mail is an application that sends digital message as mail.
Email use the \texttt{Simple Mail Transfer Protocol} (SMTP). 

\subsection{Computer Aided Design}
CAD is a type of software that assists engineers is creating a design.
CAD engineers to inspect a design from many positions or angles, allowing them to create designs better.

\subsection{Graphics Processing Software}
A graphic processing software allows designers to edit, create, crop, erase, an image.

\section{Common Features of an Application}
The two typical applications in a computer system is a GUI and a CLI.

A Command Line Interface (CLI) is an application than runs in a terminal.
A CLI is faster, use less memory, and more powerful.
Yet, it is harder to learn.

A Graphical User Interface (GUI) is an application that utilize icons and pictures to make the application more accessible.
It is easy to use compared to a CLI. 
But requires more memory, a graphical display and a mouse, and is more difficult to implement.

\subsection{GUI Implementation}
A common GUI includes several elements: 
\begin{itemize}
	\item Toolbar
	\item Menu
	\item Dialogue box
\end{itemize}

\section{Binary Representation}
Computer systems use bits and bytes to store information.
This is a binary system.

There is 8 bits in a byte.
A bit is a boolean, either yes or a no.
Or more accurately, a bit is a digit in binary.

The number system we commonly use is base-10.
This means for every digit incremented, the number will add by $10^n$.
Binary is base-2, where every digit incremented, the number will add by $2^n$.

\subsection{Binary}
Here's a table showing the multipliers of an eight bit binary:
\begin{tabular}{|l|c|c|c|c|c|c|c|c|}
	\hline
	Digit      & 8     & 7     & 6     & 5     & 4     & 3     & 2     & 1 \\
	\hline
	Power of 2 & $2^7$ & $2^6$ & $2^5$ & $2^4$ & $2^3$ & $2^2$ & $2^1$ & $2^0$ \\
	\hline
	Multiplier & 128   & 64    & 32    & 16    & 8     & 4     & 2     & 1
	\hline
\end{tabular}

\subsection{Negative Numbers In Binary}

There is two methods of indicating a negative number: \texttt{two's complement} and \texttt{}

\end{document}
