\documentclass[../notes.tex]{subfiles}

\begin{document}
\section{Systems in Organizations}

\subsection{System life cycle}
System life cycle discusses the process of developing a new system.

\begin{itemize}
	\item Existing System Analysis
	\item Requirement Specification
	\item Software Design
	\item Software Implementation
	\item Testing \& Debugging
	\item New System Installation 
	\item Maintenance
\end{itemize}

Often times, we can simplify this to the \texttt{Software Life Cycle}
\begin{itemize}
	\item Planning and Analysis
	\item Design Overview
	\item Development
	\item Evaluation
\end{itemize}

\subsection{Planning a New System}
We can use the acronym \texttt{TELOS} to plan whether or not to use a new system.

\vspace{1em}

\noindent \textbf{Technical Feasibility}: Can we actually implement this system? \\
\noindent \textbf{Economic Feasibility:} Is the system cost effective? \\
\noindent \textbf{Legal Feasibility:} Is there any legal issues? \\
\noindent \textbf{Operational Feasibility:} Is the current practices sufficient to support the new system \\ \noindent \textbf{Schedule Feasibility:} How long do we need to wait? \\

\subsection{Change Management}
\texttt{Change Management} refer to the process of shifting individuals, teams, departments, and organizations.
The purpose of Change Management is move a company into a more efficient state, while minimizing the negative effects of the process.

\subsection{Compatibility Issue}
There may be compatibility issues when encountering a \texttt{Legacy System}.
A Legacy System is a general term referring to old technology.
For example, an old printer may be called a legacy printer.
Keeping around legacy systems is troublesome, since companies may no longer support the technology (replacement parts are no longer produced, software is incompatible).
There may also be compatibility issues when interacting with legacy systems, as it might no support new protocol, or have a outdated proprietary software.

\subsubsection{Four Strategies of Integration}
\begin{enumerate}
	\item Keep both information system, and develop them to have the same functionality (high maintenance cost)
	\item Replace both information systems with a new one (increased inital cost)
	\item Select the best information system from each company and combine them (it is very difficult for the employees to work information systems from another company)
	\item Select one company's information systems and drop the other companies' (policy problems)
\end{enumerate}

\subsection{SaaS}
Nowadays, there is many different methods of providing business software.
Some clients install their software on their own premises, while other have the software installed on dedicated servers that provide these services.

SaaS is one of the methods of providing services.
Software and data is hosted on a remote data center, where clients pay fees to access the services.
SaaS relies heavily on the Internet.
Some examples of SaaS are Dropbox, Gmail, and Salesforce.

Nowadays, SaaS is less expensive and have a lower initial cost.
It is easier to install, maintain, and upgrade. 
It is easily scalable, allowing for rapid growth.
It requires less IT staff to manage.

However, SaaS requires a constant Internet connection.
It is based on a monthly subscription.
Since the information is stored on a remote server, there is no guarantee of security.

\subsection{Alternative Installation Process}
Installing a new system is a necessary evil in many enterprises.
Many users do not like to adjust to a new system, and changeover can cause many compatibility issue.
Yet, installing new software is an essential part of the everyday operation of the IT Department.

There are four common types of software changeovers:

\subsubsection{Parallel}
In a parallel changeover, both software work simultaneously for a period of time.
If the new system fails, the old system is usable.
Only when the new system have been running smoothly can the old system be retired.

Parallel changeover is the safest method, but the most time consuming.
As both system is required to be running, additional resources is needed to support both systems.
However, this method is not effective if the old system and new system does different tasks.

\subsubsection{Big Bang}
Big Bang, or direct changeover is a risk operation where the old system is retired immediately, and the new system is immediately put in use.
This system is risky, and should be only used for non critical systems.
In this case, all users need to be trained before-hand on the new system.

\subsubsection{Pilot}
A pilot changeover is used in a company with a lot of sites.
In a pilot changeover, a new system is introduced in sites one at a time.
This method allows for a enterprises to test out the new system before implementing it fully.
And the feedback received from each system can be used to improve the changeover at other sites.

\subsubsection{Phased}
A phased conversion is used a system with many modules. 
A company may change one module at a time, so that the system is converted at different times.
This approach allows for an extended training an adoption of the system, but takes more time.

\subsection{Data Migration}
Data migration occurs when data in transfered between different storage devices or different formats.
There are many problems that will occur when data is transfered between two systems. 
Errors such as incompatibility, corruption, virus, etc, can cause irreparable lost to the data.
To guarantee the safety of the data, following three steps when migrating data.

\begin{enumerate}
	\item Plan
	\item Migrate
	\item Validate
\end{enumerate}

\end{document}
