\documentclass{standalone}

\begin{document}

\section{Human Resources Planning}

\subsection{Human Resources Planning}
Work force planning is the method used by a business to forecast how many and what type of employees are needed now and in the future.
Human Resource department is often in charge of this. 

There is two types of workforce planning: short term and long term planning.
Short term planning work on fulfilling immediate needs such as current vacancies.
Long term planning work on planning to accommodate strategic change in the organization.

There is two major activities in terms of workplace planning.
The first task is the gathering of information HR needs of a company. 
This means data about the company's population, turnover rates (refer to~\ref{LaborTurnover}), employee shortages, and strategic goals.  The second activity is developing the response HR need to take.

A work force plan should follow the steps below:
\begin{itemize}
	\item Assess current employees
	\item Analyze demand for employees
	\item Analyze the supply of employees
	\item Compare the demand and supply data
	\item Develop and implement the work force plan
\end{itemize}

\subsubsection{Forecasting the demand for employees}
To predict the future employee need of a firm, we would use several different approaches:
\begin{itemize}
	\item Past Data
	\item Productivity of Workers
	\item Management Knowledge
	\item Calculating staff turnover
\end{itemize}

However, there are many factors that can influence the demand for employees.
Several are listed below:
\begin{itemize}
	\item Change in demand of goods and services
	\item Technology developments
	\item Change in business goals
	\item change in organization structure
	\item change in production techniques
\end{itemize}
Thus, it is often worthwhile to take not just use past data to evaluate the demand for employees.
Many times, knowledge from employees and managers are just as important.

\subsubsection{Analyzing the Supply of Employees}
After forecasting the demand of employees, the company need to consider the important question: how can we find the right employees?
The supply of internal employees should be analyzed by looking at the number of employees and their function, their age, and other important characteristics.
The supply of external employees should be analyzed by looking at the market of recruitment.
Recruiting internally and externally have its advantages and disadvantages.

\begin{center}
\fbox{%
	\parbox{0.8\textwidth}{%
		\noindent\textbf{Internal} labor is sources of labor that is already present in a company. \\
		\noindent\textbf{External} labor is sources of labor that need to be hired.
	}
}
\end{center}

\subsubsection{Internal Employees Supply}
This will likely depends on the company's stance on:
\begin{itemize}
	\item Internal Promotion
	\item Staff Development
	\item Labor Turnover
	\item Legal conditions for redundancy and dismissal
\end{itemize}

\subsubsection{External Employee Supply}
The business need to analyze a range of local and national factors:
\begin{itemize}
	\item Housing and Transport
	\item Level of Competition
	\item Rate of Unemployment
	\item Government Training and Subsidies
	\item Skill available in the region
	\item Population and demographics
	\item Cost of recruitment
\end{itemize}

\subsection{Labor Turnover}\label{LaborTurnover}
Labor Turnover is the percentage of the workforce that leaves every year.
There are many reasons in which an employee may leave:
They can be fired, sick, redundant.

To measure employee turnover, we use a equation:
\begin{equation}
	\texttt{Labor Turnover} = \frac{\texttt{Number of Employees Leaving}}{ \texttt{Number of Employees in the business} } \cdot 100 \\
\end{equation}

Different types of company have different labor turnover rates. 
However, in general, the higher the labor turnover, the more troubling it is.

\subsubsection{Avoidable Causes of Employee Leaving}
There are some avoidable causes for employee leaving:
\begin{itemize}
	\item Dissatisfaction with payment
	\item Poor Working Environment
	\item Job Dissatisfaction
	\item Human Resource Policies
	\item Lack of Facilities
	\item Dissatisfaction with Working Time
\end{itemize}

\subsubsection{Unavoidable Cause of Employees Leaving}
\begin{itemize}
	\item Family Circumstance
	\item Physical Reasons
	\item Marriage
	\item Birth of Children
	\item Retirement
	\item Dismissal
	\item Redundancy
\end{itemize}

\subsubsection{Cost of Labor Turnover}
\begin{itemize}
	\item Recruitment
	\item Loss of Productivity
	\item Inefficiency
	\item Training
	\item Reputation of Company
\end{itemize}

There are numerous disadvantage of a high Labor Turnover.
Thus, it is in the best interest of the company to keep Labor Turnover low.

\subsection{Internal and External Factors}

Factors that can affect workforce planning can be grouped into two factors: \textbf{External} and \textbf{Internal} factors.

\subsubsection{External Factors}
\begin{itemize}
	\item Competition
	\item Payment
	\item Legislation
	\item Technology Advancements
	\item Population and Demographics
	\item Economic Situation
	\item Availability of Skills
	\item Government Training and Subsidies
\end{itemize}

\subsubsection{Internal Factors}
\begin{itemize}
	\item Budget
	\item Policy on Promotion
	\item Working Practice
\end{itemize}

\subsection{Recruitment Process}
The recruitment process is very important.
A bad recruitment process may lead to a high Labor Turnover.
A common recruitment process looks like this: 
\begin{enumerate}
	\item Job Analysis
	\item Job Description
	\item Person Specification
	\item Job Evaluation
	\item Job Advertisement
	\item Selection
\end{enumerate}

\subsubsection{Job Analysis}
Job Analysis looks at what the job entails.
Such as what skills are required, how much training is needed, what are the tasks needed completion.

\subsubsection{Job Description}
Job description is a way to \textit{pitch} your job.
To explain the basic picture of a job to the candidates.

\subsubsection{Person Specification}
Person specification defines the qualities of an individual in order to take the job.
These specifications can be skills or culture.

\subsubsection{Job Evaluation}
Job Evaluation allows for managers to decide the value of the job.
This sets the pay for the job.

\subsubsection{Job Advertisement}
This is when a company promote a job, internally or externally.
Internally, this would be a promotion.
Externally, this would be an advertisement.

\subsubsection{Selection}
This would be the stage where the candidate is chosen.
Often, there are different ways to carry out this stage.
Some commonly used methods will be listed below:
\begin{itemize}
	\item Application
	\item Job Interview
	\item Testing
	\item Job Offer
\end{itemize}

\subsection{Internal and External Recruitment}

\subsubsection{Internal Recruitment}
There are several advantages and disadvantage of Internal Recruitment.

\textbf{Advantages}\\
\begin{itemize}
	\item Shorter training
	\item Reuse existing resources
	\item Managers are familiar with employee
	\item Cheaper
	\item Motivating to other employees
	\item Retains valuable employees
\end{itemize}

\textbf{Disadvantages}\\
\begin{itemize}
	\item Limited applicants
	\item Limit to better employees outside
	\item Vacancy is created, and need to be filled
	\item Less new ideas are brought into an organization
	\item May cause conflict in organization
\end{itemize}

\subsubsection{External Recruitment}
There are several advantages and disadvantage of External Recruitment.

\textbf{Advantages}\\
\begin{itemize}
	\item Brings new idea to the table
	\item Encourage existing staff to update their skills
	\item Promotes change
	\item Offer greater choice
\end{itemize}

Disadvantages\\
\begin{itemize}
	\item New employees may not fit in
	\item Demotivated existing staff
	\item Costly and time consuming
	\item Require training
	\item Higher risk of unsuitable employment
\end{itemize}

\subsection{Types of Training}
Training is very important for a business.
It brings many benefits, but is often overlooked.

\subsubsection{Benefits of Training}
\begin{itemize}
	\item Increase productivity
	\item Can replace other in an organization restructure
	\item Catch up to new technology
	\item Increase employee confidence
	\item Increase job satisfaction
	\item Increase chance of promotion
	\item Give a competitive advantage to the company
\end{itemize}

\subsubsection{Need for Training}
There are three levels of training needs.

\textbf{Organization Level}: A change in the company structure may require the employees to trained to achieve new objectives.

\textbf{Department Level}: Some employees may need to be trained due to specifics reasons. 
Such as absences, production levels, or customer complaints.

\textbf{Individual Level}: Some employee may request additional training themselves.
Perhaps to keep up to date with the industry, or to support a new role.

\subsubsection{Cost of Training}
Other than the expenses of teaching an employee, managers would also need to consider the time taken away from the job.

\subsubsection{On the Job Training}
On the job training is a type of training given to the new employee during work.

\textbf{Induction Training}: Training at the start of a job.

\textbf{Coaching}: Supervisor guides an employee through a process.

\textbf{Mentoring}: Employee is paired with a more experienced worker.

\textbf{Job Rotation}: Employee works in different job positions in the company.

\textbf{Apprenticeship}: Trainees work under supervision for a long period of time

\textbf{In-house Course}: Company organized training.

\textbf{E-Learning}: Use of multi-media to teach employee new skills.

\subsubsection{Off-the-job Training}
Off the job training minimizes distraction, but the skills learned may not relate to the job as closely.

\textbf{Lecture and Conferences}: Verbal presentation involved with a large audience.

\textbf{Vestibule Training}: Training employees using a prototype work environment.

\textbf{Simulation}: Training involving specialized equipment that simulates working environment.

\textbf{Case Studies}: Use existing examples and stories to discuss with employees.
This often focuses on decision making.

\textbf{Role-playing}: Trainees play a role without rehearsal. 
This often focus on inter person relationships.

\subsubsection{Advantage vs Disadvantages}
On the job training is cheaper and more related to the work.
However, it disrupts the work flow of both the employee and the mentor.

Off the job training is more organized, but less relevant to the job.
It is also more expensive.

\subsection{Types Appraisal}
\begin{itemize}
	\item Performance appraisal by supervisor
	\item Appraisal by a manager up in the hierachy
	\item Formative appraisal
	\item Summative appraisal
	\item Self-appraisal
	\item 360-degree appraisal
	\item Management by objectives
\end{itemize}

\subsection{Dismissal and Redundancy}

\subsubsection{Lawful reasons for dismissal}
\textbf{Misconduct}: cases such as violence, discrimination, theft, or fraud.

\textbf{Capability}: Poor performance after repeated attempts of aid.

\subsubsection{Steps to dismissal}
\begin{enumerate}
	\item Full investigation
	\item Complete check
	\item Written evidence
	\item Meeting with employee
	\item Written notice
\end{enumerate}

\subsubsection{Types of Redundacy}
\begin{itemize}
	\item Job Redundancy
	\item Work place redundancy
	\item Employee redundancy
\end{itemize}

\subsubsection{Common Redundancy steps}
\begin{enumerate}
	\item Plan redundancy
	\item Identify Alternative
	\item Prepare a schedule
	\item Inform Employee
	\item Redundancy Selection
	\item Individual Consultation
\end{enumerate}

\subsection{Work Patterns and Practices}
\subsubsection{Part-Time Work}
Employees work only part of the working time.
Often done be individuals with many jobs, or have to be working on something else (education).

Part time worker's wages are cheaper and they are easier to replace.
They get less benefits and are more flexible.
However, part time workers have less motivation and loyalty.
There is also a higher training cost.

\subsubsection{Temporary Employment}
This is when employees are employed only for a short period of time.

Temporary workers receive higher wages, and are more flexible.
But they may lack loyalty and motivation.

\subsubsection{Flextime Employment}
Employees can work with a flexible time schedule.

This allows for more employee work satisfaction.
But creates difficulty with scheduling.

\subsubsection{Teleworking}
Employees can work at a separately from the company.

This reduce cost of office space and facilities.
It increases employee satisfaction, allows more working time.
But may be detrimental to collaboration, and unsuitable for many types of jobs.

\subsubsection{Portfolio Workers}
These workers can work for many companies and offer specific, specialized work.

\subsection{Outsourcing, offshoring, and reshoring}

\subsubsection{Outsourcing}
Outsourcing is when business transfer part of its work to outside companies.

This save HR costs and overhead costs, increase specification, reduce training.
But take time to sign a contract, may be dangerous, and may have issues in communication.

\subsubsection{Offshoring}
Offshoring is when companies move their manufacturing facilities to another country.
This often cause a loss of jobs.  
However, offshoring reduce costs.

This is often done due to low wages in foreign countries and lower taxes.

\subsubsection{Reshoring}
Reshoring occurs by bring back manufacturing facilities from Offshoring.

This happens to support the home country's economy, or for CSG reasons.

\subsubsection{Nearshoring}
Offshoring, but to a county close to the home country.

\subsection{Influence of innovation, ethics and culture on work force practices}

\section{Organization Structure}

\subsection{Terminology}
\subsubsection{Delegation}
Delegation is the act of assigning responsibility to a person, in order for them to carry out a task.
Normally, one delegates downwards to someone in a lower rank.
However, the person who delegates the task is still held responsible of the task.

\subsubsection{Span of Control}
Span of Control decides the number of employees a manager can directly control.
This number varies through different situations.
Complex work requiring supervision would have a narrow span of control, while work like mass production have a wide span of control.

Narrow control means more direct supervision.
Wider control means more empowerment and delegation.
It also means better communication.

\subsubsection{Level of Hierarchy}
Levels of Hierarchy is the amounts of levels in the company hierarchy.
The larger the level of hierarchy, the more narrow the span of control.

\subsubsection{Chain of Command}
Chain of command defines the line of authority in which responsibility and orders are passed down from one person to another.
The longer the chain of command, the more time it would take to communicate up and down the layers of hierarchy.

\subsubsection{Bureaucracy}
Bureaucracy is a system in which a clear hierarchical structure and chain of command has been set up.
Employees are expected to follow the administration precisely.

\subsubsection{Centralization and Decentralization}
Decentralization means a transfer of decision-making power.
In a centralized organization, only a small group of people make all the decisions.

\subsubsection{De-layering}
The act of remove the numbers of layers in a organization hierarchy.

\subsection{Types of Organizational Charts}
A organization chart illustrates the flow of communication between the members of an organization.

\subsubsection{Flat/Horizontal}
A flat or horizontal organization chart is one that has a few levels of hierarchy, short chain of command, and wide span of control.
In this model, middle management is removed.
Employees are given more power to make decisions.
This is not a good model for large company with many projects.

\subsubsection{Tall/Vertical}
A tall or vertical organization structure have many levels of hierarchy, long chain of command, and narrow span of control.
In this model, there is more direct supervision, and less delegation.

\subsubsection{Hierarchical Organizational Structure}
A tall organization structure with many levels of hierarchy.
Looks similar to a pyramid.

\subsubsection{Organization by Product}
The business is organized into different departments focused on different products.
The departments are more specialized, but they may be duplication in these departments.

\subsubsection{Organization by Function}
The business is organized into different roles. 
For example, Finance, HR, Marketing.
This is implemented in a stable environment when it is expected that business strategies does not change often.

\subsubsection{Organization by Region}
Business is organized by region.
Often used in multi-national corporations.
This allows for more culturally focused management, but the management is decentralized.

\subsection{Changes in Organizational Structure}
\subsubsection{Project Based}
A temporary structure which is created to finish a specific project.
A project manager can manage employees from many different departments.
This allows for flexibility and scheduling, but makes the teams self-sufficient.

\subsubsection{Shamrock}
In a Shamrock organizational structure, companies keep core workers with a multitude of skills.
All other employees should be temporary or out sourced.

This increases efficiency and flexibility, but have legal issues in some environments.

\section{Leadership and Management} 
\subsection{Key Functions}
Types of managers differ depending on a corporation and their needs.
The most common are:
\begin{itemize}
	\item Planning
	\item Coordinating
	\item Commanding
	\item Controlling
	\item Organizing
\end{itemize}

\subsection{Management VS Leadership}
Management is the act of directing the many resources of a business organization to achieve business objectives.
Leadership is the use of strategic thinking and smart thinking to inspire individuals to meet challenges and accomplish goals.

A leader inspires, while a manager controls.

\subsection{Leadership Styles}
\subsubsection{Situational}
Leadership that changes depending on the situation.

\subsubsection{Paternalistic}
The leader looks after its employees as if they are family.

\subsubsection{Laissez-faire}
A hands-off leadership style.

\subsubsection{Autocratic}
Strong and rule based leadership.
Relies on following instructions and orders.

\subsubsection{Democratic}
Leadership that focus on employees' input.

\section{Motivation}

\subsection{Motivational Theory}

\subsubsection{Frederick Winslow Taylor}
Break down each task into small components, then scientifically teach the employees.
Offer extrinsic incentives in the right place.

This model ensures higher efficiency, but is suited only for simple tasks.

\subsubsection{Abraham Maslow}
Maslow's pyramid showcase the levels of needs for each individual.
Below is the layers of the pyramid in order.

\textbf{Self-Actualization}: Recognition of one's potential.
Creative self fulfillment.

\textbf{Esteem needs}: Mastery and achievement in a chosen field.

\textbf{Love and Belongingness needs}: The need to be loved the trusted. 
To below in a family.

\textbf{Safety needs}: To have stability and safety in the environment.

\textbf{Physiological needs}: To have food, shelter, clothing

Maslow argues that you need to satisfy the basic needs before the complex needs.

\subsubsection{Fredrick Herzberg}
There are two different needs.
Hygiene needs and Motivational needs.

Hygiene need, if not met, causes dissatisfaction with work.
Motivational needs, when met, causes satisfaction with work.

\subsubsection{John Stacey Adams}
An employees should receive equal inputs and outputs.
When employees are satisfied when the inputs and the outputs are perceived to be equal.

\subsubsection{Daniel H. Pink}
Pink focus on Autonomy, Mastery, and Purpose.
He believes that this is the modern way to achieve motivation.

\subsection{Financial Motivation Tools}
\begin{itemize}
	\item Salary
	\item Wages
	\item Commission
	\item Profit-based pay
	\item Performance-related pay
	\item Employee share-ownership schemes
	\item Perks
\end{itemize}

\subsection{Non-Financial Motivational Tools}
\begin{itemize}
	\item Job rotation
	\item Job enlargement
	\item Job enrichment
	\item Empowerment
	\item Opportunity to make a difference
	\item Teamwork
\end{itemize}

	
\end{document}
