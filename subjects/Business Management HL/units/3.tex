\documentclass{standalone}

% Adapted from Kognity
\begin{document}

\section{Sources of Finance}

\subsection{The Role of Finance}
There are many things companies spend money on. 
This subsection will list some of the uses of finance.

\paragraph{Capital Expenditure}
One of the most common source of expenditure is \texttt{Capital}.
Capital investment is spending money on \texttt{fixed assets}, which are assets that are hard to sell.
Generally, capital expenditures are used to fund long term goals, such as building facilities or buying machinery.
Capital expenditures are often funded by long term sources of finance.

\paragraph{Revenue Expenditure}
Revenue expenditure is spending money on general operation costs.
Expenditures such as wages or rent.
Revenue expenditures are often funded by short to medium term sources of finance.
When a firm cannot pay its revenue expenditure, they go into a state of insolvency.

\subsection{Internal Source of Finance}
Internal sources of finance refers to money collected internally.
Such as the sale of assets, retain profit, or personal funds.

\begin{description}
    \item[Personal Funds] is the money invested by the owners of the business
    \item[Retained Profit] is profit leftover (after paying the bills) from end of a trading year
    \item[Sale of Assets] is money gained from selling any assets
\end{description}

\subsection{External Source of Finance}
Opposite to Internal Sources of Finance, External Sources of Finance is money gathered through outside means.

\subsubsection{Equity Finance}
Equity Finance is money gathered by selling ownership of the company.
Equity does not come with any interest, or requirement of repayment.
However, selling equity comes with the cost of losing control and dividends.

\begin{description}
    \item[Share Capital] is money raised through selling shares
    \item[Business Angel] are wealthy investors, often buying a chunk of shares as investment
    \item[Venture Capitalist] are companies similar to Business Angels
\end{description}

\subsubsection{Debt Finance}
Debt finance is the method of borrowing money to acquire finance.
Often times, borrowing money can quickly bring money for investment, but their is the cost of interest.
Interest is additional money owed overtime as borrowed money is payed off.

\begin{description}
    \item[Loan Capital] are long term borrowing of money, often for the purchasing capital. These loans require collateral in case there is a default on the loan.
    \item[Overdrafts] are a high cost, short term loan. It is when the company spend more money that they have in their account, and have to pay back in high interest.
    \item[Credit Cards] are a method of borrowing money and paying back every month.
\end{description}

\subsubsection{Financial Aid}
Financial Aid is money given to the companies for free.
Generally, these come from NGOs or governments who want to support the business.

\begin{description}
    \item[Subsidies] money given to the production of goods that is good for the society, often provided by the government
    \item[Grants] are loans with no interest, and does not need to be paid back. There may be conditions on how the money is spent
\end{description}

\subsubsection{Other Sources of Finance}
Other then the main sources of finance, there are others that does not fit the groups.

\begin{description}
    \item[Trade Credit] is a method of paying for goods at a later date, without interest. Often provided to companies by companies.
    \item[Debt Factoring] is the action of selling debt, to a debt factoring company. Often at a lower cost, but with an immediate payment.
    \item[Leasing] is the action of leasing fixed assets instead of buying them. Flexible, but cost more in the long run.
\end{description}

\subsection{Short, Medium, and Long-Term Finance}
External sources of finance can be broken into 3 types according to their durations.
Internal sources of finance can be fall into any of these categories.

\subsubsection{Short-Term}
Short-term finance are repaid within 12 months. 
These are normally used to solve cash flow problems or to pay for revenue expenditure.
Short-term finance is often expensive and have high interest.

Below is a list of short term finances:
\begin{itemize}
    \item Overdraft
    \item Trade Credit
    \item Debt Factoring
    \item Leasing
    \item Subsidies
\end{itemize}

\subsubsection{Medium-Term}
Medium-term finance last longer than 12 months, but less than 5 year.
These are normally used for buying fixed assets or capital.
Medium term finance is in-between of short and long term finance.

Below is a list of medium term finances:
\begin{itemize}
    \item Loan Capital
    \item Leasing
    \item Subsidies
\end{itemize}

\subsubsection{Long-Term}
Long term finance last longer than 5 years.
Mortgages and all equity finance belong in this category.

Below is a list of long term finance:
\begin{itemize}
    \item Share Capital
    \item Venture Capital
    \item Business Angel
    \item Loan Capital
    \item Grants
\end{itemize}

\subsection{Evaluation of Sources of Finance}
Often times, companies have to make decision on which source of finance to choose.
Different methods of finance have different purposes, with different opportunity cost and effectiveness.

There is three general idea for choosing sources of finance, \texttt{gearing}, \texttt{purpose}, and \texttt{ownership}.

\subsubsection{Gearing}
\textit{This will be further explained in 3.6, Efficiency Ratio Analysis.}

Gearing ratio calculates the percentage of loan capital versus the total capital of the business.
Having a high gearing ratio makes the company risky in case they default on their loans.
However, having a high gearing ratio lowers the amount of ownership that needs to be split.

\subsubsection{Purpose}
Consider the purpose of the funds when gathering finance.
Determine if the source of finance falls into short, medium, or long term groups.

\subsubsection{Ownership}
Different companies have access to different kinds of finance.
Sole traders and Partnerships have access to mostly internal sources of finance.
They can also take loans and use trade credit.
Bigger corporation generally cannot use personal funds, but can take advantage of equity for financing.

\section{Cost and Revenue}
In order for a business to make money, its revenue must be larger than its cost.
In this section, we will discuss the different cost and revenue of a company.

\subsection{Classification of Costs}
Cost can be split up into two major categories, Fixed Cost and Variable Costs.

\subsubsection{Variable Cost}
Variable cost is cost that change directly due to production.
These include material and labor used for production.

\subsubsection{Fixed Costs}
Fixed cost is cost that is not effected directly with production.
For example, electricity bills remain relatively constant no matter the amount produced.
Although electrify is required to make the product, it is not effected directly by the amount produced.

Other fixed costs include rent, salaries, capital.

\subsection{Direct and Indirect Costs}
Costs can also be clarified into direct and indirect costs.
Direct costs are those that directly impact the good and service a company produces.
For example, raw material is a direct cost, but coffee machine refills are an indirect cost.

Examples of direct costs are salary or utility, while examples indirect costs are infrastructure cost and advertising.

\section{Revenue and Revenue Streams}
Revenue is the money earned from the selling of goods and services.
It is different than profit, in that profit is revenue minus cost.
Simply, revenue can be calculated by the selling price of a product times the amount sold.

Revenue streams are methods of generating revenue.
Many companies do not earn money from one source.
For example, a movie theater do not earn money just from selling tickets, but also food and drinks.
Or newspapers that do not earn money just from selling newspapers, but also from advertisements.

\section{Break-Even Analysis}
In order to maintain a health revenue and cost relationships, many companies use break even analysis plan ahead.
Break even analysis allows a company to find how much it need to sell in order to \texttt{break even} (profit is greater than zero).

\subsection{Calculating the Break-Even Point}
There are several units used in the calculation that have to be noted.

\begin{description}
    \item[Fixed Cost] is cost not effected by change of production
    \item[Variable Cost] is cost directly effected by change of production
    \item[Contribution per Unit] is the price per unit minus the variable cost per unit
    \item[Break Even Point] is the unit of product sold that the total profit is zero
    \item[Margin of Safety] is the difference between the current unit sold and the break even point
\end{description}

The break even point can be calculated using the following equation:
\begin{equation}
    \frac{\textrm{Fixed Costs}}{\textrm{Contribution per Unit}} 
\end{equation}

\subsection{Break-Even Charts}
Break even chart shows the break even point against unit sold and profit.
The x axis is output, or the amount sold.
The y axis is revenue and cost.

There are three lines.
The first line is the Fixed Cost, which is a parallel line with equation $y=t\texttt{fixed cost}$.
The second line is the Total Cost, which starts at the fixed cost, and increase over time.
It's equation is $y = \texttt{variable cost per unit} \cdot x + \texttt{fixed cost}$.
The last line is the Total Revenue, which starts at zero.
It has the equation $y = \texttt{contributions per unit} \cdot x$

The point at which Total Revenue intersects Total cost is the break even point.
Draw a line from the Break even point to the x axis.
The difference between the break even point and the actual output is the margin of safety.

\subsection{Evaluation of Break-Even Analysis}
Break even analysis is easy to use, and show the important information that keeps the company alive.
A company can use break even analysis predict how changes in production can change the profit levels.
However, break even analysis is not very accurate. 
It ignores a companies' inventory and economies of scale. 
It assumes all products are sold at the same price.

\end{document}
