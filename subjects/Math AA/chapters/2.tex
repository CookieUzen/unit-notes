\documentclass[../notes.tex]{subfiles}

\begin{document}

\section{Functional Relationships}
The set of x-values that make up a function is called the domain.
The set of y-values that make up a function is called the range.

An equation is a function only when $f$ acts on all elements of the domain, and each domain pairs with one and only one range element.
This test is called the vertical line test.
If one were to drawn an infinitely long vertical line on any x domain, then the vertical line will only intersect the function at one point.

There is a similar test called the horizontal line test. 
In which a infinitely long horizontal line is drawn on any $y$ range, then the horizontal line will only intersect the function at one point.
A function must pass the horizontal line test in order to have an inverse.

\subsection{Odd and Even Functions}
An even function is a function that is symmetric over the $y$ axis.
Or in other words, $f(x) = f(-x)$.

An odd function is a function that is symmetric over the line $y=-x$.
Or in other words, $f(x) = -f(-x)$

\subsection{Mapping}
A one to one function is where every point of the domain corresponds to one point on the range.
A one to many function is where every point of the domain corresponds to many points on the range, this if not a function.
A many to one function is where many points of the domain corresponds to the same point on the range, this function does not have an inverse function.
A many to many function is where many points on the domain corresponds to many points on the range, this is not a function.

\subsection{Self Inverse}
If the inverse of a function is itself, it is a self inversing function.
\begin{equation}
	f^{-1}(x) = f(x)
\end{equation}

\section{Special Functions and Their Graphs}

\subsection{Domain and Range of Common Functions}

\begin{alignat}{3}
	y &= mx + b \qquad &&x \in R,\; y \in R\\
	y &= ax^2 + bx + c \qquad &&x \in R,\;
	\begin{dcases}
		\textrm{if } a > 0, y \ge k \\
		\textrm{if } a < 0, y \le k
	\end{dcases} && \qquad \textrm{where } (h,k) \textrm{ is the vertex}\\
	y &= a^x \quad (a>0,\; a\neq1) \qquad && x \in R,\; y > 0 \qquad && (y = 0 \textrm{ is an asymptote}) \\
	y &= \log_a{x} \quad (a>0,\;a\neq1) \qquad && x > 0,\; y \in R \qquad && (x=0 \textrm{ is an asymptote}) \\
	y &= \frac{1}{x} \quad && x \neq 0,\; y \neq 0 \qquad && (x=0 \textrm{ and } y=0 \textrm{ are the asymptote}) \\
	y &= \frac{ax+b}{cx+d} \quad && x \neq -\frac{d}{c},\; y\neq \frac{a}{c} \\
	y &= \sqrt{x} \quad && x \ge 0,\; y \ge 0 \\
	y &= \sin{x} \quad && x \in R, \; -1 \le y \le 1 \\
	y &= \cos{x} \quad && x \in R,\; -1 \le y \le 1 \\
	y &= \tan{x} \quad && x \neq \frac{\pi}{2}+k\pi,\;k \in Z,\quad y \in R \\
\end{alignat} 

In general, for the domain of each equation, watch out for vertical asymptotes and undefined numbers ($\sqrt{-1},\; \frac{1}{0}$).

For the range of each equation, watch out for horizontal asymptotes, turning points/vertex, and domain limitations.

\subsection{Quadratic Functions}

\subsubsection{Forms of Quadratics}
A quadratic function have 3 forms: Standard form, Vertex form, and Intercept form.

\begin{align}
	y &= ax^2 + bx + c \\
	y &= a(x-h)^2 + k \\
	y &= a(x-p)(x-q) 
\end{align} 

In the standard form, $a$ defines to concavity of the function.
If $a>0$, the quadratic concaves up, if $a<0$, the quadratic concave down.
$c$ represents the x-intercept of the quadratic.

In vertex form, $(h,k)$ is the vertex of the quadratic.
A vertex is the turning point of the quadratic, it is also where the axis of symmetry lies.
The quadratic is mirrored across the axis of symmetry.

In the intercept form, $p$ and $q$ are the x intercepts (or zeros) of the function.

\subsubsection{Equations}
The equation for the axis of symmetry is:
\begin{equation}
	x = -\frac{b}{2a}
\end{equation}

The sum of roots of a quadratic is:
\begin{equation}
	(\alpha+\beta) = -\frac{b}{a}
\end{equation}

The product of roots of a quadratic is:
\begin{equation}
	(\alpha \cdot \beta) = \frac{c}{a}
\end{equation}

\subsubsection{Finding the Roots}
There are four ways to find the roots of a quadratic.
You can graph it using the GDC.
You can use the quadratic formula.
You can use complete the square.

The quadratic formula is listed below:
\begin{equation}
	x = \frac{-b \pm \sqrt{b^2-4ac}}{2a}
\end{equation}

$\Delta = b^2-4ac$ is the discriminant.
If:
\begin{itemize}
	\item $\Delta > 0$, the function have 2 real roots
	\item $\Delta = 0$, the function have 1 unique real root
	\item $\Delta < 0$, the function have no real roots. 
\end{itemize}

To factor, simply factor the quadratic into the intercept form using the sum and product of roots equations.

To complete the square, start with the equation in standard from.
Set $y$ to zero.
Then, remove the coefficient of x (by factoring it out).
Afterwards, use the property $(a+b)^2  = a^2 + 2ab + b^2$ to factor the equation into vertex form.

\subsection{Rational Function}
A function is rational function if its is made up of the dividing of two polynomials.
\begin{equation}
	y = \frac{f(x)}{g(x)}
\end{equation}
Where $f(x)$ and $g(x)$ are the two polynomials.

The reciprocal function $y=\frac{1}{x}$ is a polynomial.

There is four important points on a polynomial. 
These are the:
\begin{itemize}
	\item Horizontal Asymptote
	\item Vertical Asymptote
	\item X-Intercept
	\item Y-Intercept
\end{itemize} 
We generally use plot these points/lines first before drawing a polynomial.

To find the horizontal asymptote, there is 3 possibility.
\begin{itemize}
	\item If $f(x)$ has the higher degree, then there is no horizontal asymptote
	\item If $g(x)$ has the higher degree, then the horizontal asymptote is $y=0$ 
	\item If $g(x)$ and $f(x)$ have the same degree, then the horizontal asymptote is $y=\frac{f_n}{q_n}$, where $f_n$ and $q_n$ are the leading coefficient of  $f(x)$ and  $p(x)$. 
\end{itemize}

The vertical asymptote is when $g(x) = 0$.
Thus, find the roots of $g(x)$.

To find the x-intercept, find the roots of $f(x)$.

The y-intercept is at point $\frac{b}{c}$.
Where $b$ is the constant of $f(x)$ and  $c$ is the constant of $g(x)$. 

\subsubsection{Drawing Rational Function}
After plotting all four points/lines, plot a sign diagram.
Where the roots of $f(x)$ and $g(x)$ are the numbers.
This sign diagram represents the domain of the rational function.
Find whether the value of the y is positive or negative in each section of the sign diagram.

Use this information to plot the rational function.

\subsection{Absolute Value Functions and Equations}
\begin{equation}
	|a| = 
	\begin{dcases}
		\;\;\;a,\; a \ge 0 \\
		-a,\; a \le 0 \\
	\end{dcases}
\end{equation}

\subsubsection{Useful Features of An Absolute Function}
\begin{align}
	|a| &\ge 0 \\
	|-a| &= |a| \\
	|ab| &= |a||b| \\
	|a+b| &\le |a| + |b| \\
	|a-b| &\ge |a| - |b| \\
\end{align} 

\subsubsection{Removing Absolute Value}
You can add a $\pm$ on the other side. 
\begin{align*}
	|x| &= y \\
	x &= \pm y
\end{align*} 

Or square both sides:
\begin{align*}
	|x| &= y \\
	x^2 &= y^2
\end{align*}
Note, squaring changes the domain. 
Recheck you answers after squaring.

\subsubsection{Inequalities}
You cannot add $\pm$ in a inequality.
As $\pm$ means potentially flipping the inequality.
Thus, we split the equation up into two parts.

\begin{align*}
	y &> |x| \\ 
	y < x \; &\textrm{ or } \; y > x \\
\end{align*} 

There are some short cuts.
\begin{align}
	x < |a| &= -a < x < a \\
	x > |a| &= x > a \textrm{ or } x < -a 
\end{align} 

\section{Operations with Function}
To do functional operation, put the function in quotes, then do operations.

\begin{align*}
	f(x) &= x+5 \\
	g(x) &= 3x \\
	f(x) \cdot g(x) &= (x+5)(3x)
\end{align*} 

\subsection{Composite Functions}
\begin{align*}
	f \circ g(x) = f(g(x))
\end{align*} 
In order for the composite function exist, the range of $g(x)$ has to fit within the domain of $f(x)$. 
Else, some values may be undefined.

The range of $f \circ g(x)$ is a subset of $f(x)$.

\subsection{Identity Function}
The identity function is a function which when composed with another function, it remains unchanged.
\begin{equation}
	f \circ g(x) = f(x)
\end{equation}
$g(x)$ is an identity function.

Generally, $f(x) = x$ is an identical function.

\subsection{Inverse}
An inverse function is a function that returns the identity function when composited with another function.
\begin{equation}
	f(x) \circ g(x) = x
\end{equation}
We often write this function in special notation: $f^{-1}(x)$

To find the inverse function of a certain function, swap the x and y values of the function.
Then solve for y.
\begin{align*}
	f(x)    & = \frac{4x}{2x-2} \\
	x       & = \frac{4y}{2y-2} \\
	2yx -2x & = 4y \\
	2yx-4y  & = 2x \\
	y(2x-4) & = 2x \\
	y       & = \frac{2x}{2x-4} \\
	f^{-1}(x) & = \frac{2x}{2x-4}
\end{align*} 

Note, when calculating the point where the inverse of a function intersect with the function, we can use the equation: $f(x) = x$ to find the point of intersection.

The domain of the inverse function is the range of the original function. 
The range of the inverse function is the domain of the original function.

\subsection{Reciprocal Function}
The reciprocal function is 
\begin{equation}
	f(x)=\frac{1}{x}
\end{equation}

When graphing the reciprocal of a function, there are several points important points.
\begin{itemize}
	\item all points at $y=1$ stays unchanged
	\item roots of the original function becomes asymptote
	\item positive numbers stays positive, negative numbers stay negative
	\item $y$ values above $|1|$ is stretched upwards, while $y$ values below $|1|$ shrinks
\end{itemize}

\section{Transformations}
There is three types of transformation. 
Translation, stretch, and reflection.

\begin{equation}
	f(x) \Rightarrow A \cdot f(Bx + C) + D
\end{equation} 
Where:
\begin{itemize}
	\item $A$ is vertical stretch.
	\item $\frac{1}{B}$ is horizontal stretch.
	\item $-C$ is horizontal translation.
	\item $D$ is vertical translation.
\end{itemize}

Sometimes, we would use a translation vector:
\begin{equation}
	\begin{pmatrix}
		a \\
		b
	\end{pmatrix} \rightarrow f(x-a)+b
\end{equation}

\subsection{Translation}
\begin{alignat}{2}
	&f(x+a) \qquad && f(x) \textrm{ shifts to the left by } a \textrm{ units } \\
	&f(x-a) \qquad && f(x) \textrm{ shifts to the right by } a \textrm{ units } \\
	&f(x)+a \qquad && f(x) \textrm{ shifts upwards by } a \textrm{ units } \\
	&f(x)-a \qquad && f(x) \textrm{ shifts downwards by } a \textrm{ units }
\end{alignat}

Be very careful, as adding inside the bracket will shift the function to the \textit{left}.

\subsection{Stretch}
\begin{alignat}{2}
	&f(ax) \qquad && f(x) \textrm{ is horizontally stretched by a factor of } \frac{1}{a} \\
	&f(\frac{x}{a}) \qquad && f(x) \textrm{ is horizontally stretched by a factor of } a \\
	&a \cdot f(x) \qquad && f(x) \textrm{ is vertically stretched by a factor of } a \\
	&\frac{1}{a} \cdot f(x) \qquad && f(x) \textrm{ is vertically stretched by a factor of } \frac{1}{a} \\
\end{alignat}

Be very careful, as horizontal stretching by a factor of $a$ is actually multiplying by a factor of $\frac{1}{a}$.

\subsection{Reflection}
Multiplying the contents of a function by $-1$ can reflect the function across the $x$ or $y$ axis.

\begin{equation}
	\begin{dcases}
		f(x) \rightarrow f(-x) \qquad f(x) \textrm{ is reflected by y-axis } \\
		f(x) \rightarrow -f(x) \qquad f(x) \textrm{ is reflected by x-axis }
	\end{dcases}
\end{equation}

\subsection{Modulus}
Similar to reflection, adding an absolute value reflects all negative values across an axis.
\begin{equation}
	\begin{dcases}
		f(x) \rightarrow f(|x|) \qquad f(-x) = f(x) \\
		f(x) \rightarrow |f(x)| \qquad -f(x) = f(x)
	\end{dcases}
\end{equation}

\section{Systems of Equations}
A system of equation means that there is multiple variables and multiple equations.
A system of equation with n variables will need n equations to solve completely.

There is two ways to solve a system of equation: substitution or elimination.
In a systems of equation, combine two equations to substitute or eliminate an equation.
Keep repeating the step above until you reach one answer.
Use that answer to plug in, and solve for other variables.

For system of equation with more than 2 equations, use the Gaussian Elimination method.
First, put the system of equation in matrix form.
Then manipulate the matrix until it become an identity matrix.
You can multiply each row by the same factor, and you can add a row by another row.

\begin{align}
&\begin{dcases}
	2x + y + z = 7 \\
	x + 3y + 2z = 13 \\
	4x - 2y + 3z = 9
\end{dcases} \\ 
&\left[ \begin{array}{ccc|c}
	2 & 1 & 1 & 7 \\
	1 & 3 & 2 & 13 \\
	4 & -2 & 3 & 9 
\end{array} \right] \\
&\left[ \begin{array}{ccc|c}
	1 & 3 & 2 & 13 \\
	0 & 5 & 3 & 19 \\
	0 & 14 & 5 & 43 
\end{array} \right] \\
&\left[ \begin{array}{ccc|c}
	1 & 3 & 2 & 13 \\
	0 & 5 & 3 & 19 \\
	0 & 0 & 1 & 3
\end{array} \right] \\
&\left[ \begin{array}{ccc|c}
	1 & 3 & 0 & 7 \\
	0 & 1 & 0 & 2 \\
	0 & 0 & 1 & 3
\end{array} \right] \\
&\left[ \begin{array}{ccc|c}
	1 & 0 & 0 & 1 \\
	0 & 1 & 0 & 2 \\
	0 & 0 & 1 & 3
\end{array} \right] \\
&\begin{dcases}
	x = 1 \\
	y = 2 \\
	z = 3 \\
\end{dcases}
\end{align} 

\end{document}
