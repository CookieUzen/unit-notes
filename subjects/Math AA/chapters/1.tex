\documentclass[../notes.tex]{subfiles}

\begin{document}

\section{Sequence, Series, and Sigma Notation}
A sequence is a list of number that is written in a defined order.
A sequence usually follows a predefined list of rules.
The numbers of the sequence are called terms.
Sometimes, the sequence is also referred to as a progression.

The Sigma Notation is way to write the addition of a sequence.
Sigma notation is written below, $a,b,c$ are the variables.

\begin{equation}
	\sum_{n=a}^{b} c
\end{equation}

$a$ represents the starting value of n.
$b$ represents the largest value of n.
$c$ is the equation that is to be added.

\section{Arithmetic and Geometric Sequences and Series}
\subsection{Arithmetic Sequence}
An arithmetic sequence is a series of numbers in a linear relationship.
Modeled by the equation:
\begin{equation}
	U_n = U_1 + (n-1)d
\end{equation}

Where $U_n$ represents the nth term.
$U_1$ represents the first term.
And $d$ represents the difference between each term.

The sum of an arithmetic sequence is modeled by this equation:
\begin{equation}
	S_n = n\cdot\frac{2U_1+(n-1)d}{2}
\end{equation}
This equation averages the first and last term, then multiplies the average by $n$ numbers of terms.

\subsection{Geometric Sequences and Series}
An geometric sequence is a series of numbers in a exponential relationship.
Modeled by:
\begin{equation}
	U_n = U_1 \cdot r^(n-1)
\end{equation}

Where $U_n$ is the nth term of the sequence,
$U_1$ is the first term is the sequence, 
and $r$ is the common ratio.

If the common ratio is 1, then the sequence will be a uniformed sequence.
If the common ratio is 0, then all the sequence will start with a value, and continue with all zeros.
If the common ratio is negative, then the sequence will oscillate between negative and positive.

The sum of a geometric sequence can be modeled by:
\begin{equation}
	S_n = \frac{U_1(1-r^n)}{1-r}; \; r\neq1
\end{equation}

An infinite geometric series is convergent when the sum of the series is a finite value.
If a geometric series is not convergent, it is divergent.
Generally, the common ratio of the geometric sequence have to be less that 1 but larger than 0 in order to be convergent.

The sum of a infinite convergent geometric sequence can be modeled by:
\begin{equation}
	S_\infty = \frac{U_1}{1-r}
\end{equation}

\section{Proof}
A proof is a way to mathematically validate a statement.

\subsection{Direct Proof}

A direct proof is a way of showing the truth of a given statement by constructing a series of reasoned connected established facts.
First, we identify the statement.
Then, we use axioms, theorems, to make deductions that prove the conclusion of your statement to be true.

\subsection{Proof by Contradiction}
If a logical reasoning is true, than the contrapositive is false.
First, we identify the implication of the statement.
Then assume the implication is false.
Use axioms, theorems, etc to arrive at a contradiction.
This proves that the original statement is true.


\subsection{Proof by Induction}
First, we find the starting point of a process.
This is called the basic step.
Then, we continue with the basic step until we reach the inductive step.
Finish proof.

\section{Counting Principles and the Binomial Theorem}

\subsection{Factorial Notation}
The factorial notation ($!$) is a way to denote the multiplication of a certain length of integers.
Factorial notation tends to result in very large numbers.

\begin{equation}
	n! = n \cdot (n-1) \cdot (n-2) \cdot \cdots \cdot 3 \cdot 2 \cdot 1
\end{equation}

A division of a bigger factorials by a smaller factorial can be down in the following method:
\begin{align*}
	\frac{5!}{3!} &= \frac{5 \cdot 4 \cdot 3 \cdot 2 \cdot 1}{3 \cdot 2 \cdot 1} \\
				  &= 5 \cdot 4 
\end{align*} 

\subsection{Permutation}
Permutation is written is the format of:
\begin{equation}
	\Perm{n}{k} = \frac{n!}{(n-k)!}
\end{equation}
Where $n$ is the objects to chose from and $r$ is the number of objects chosen.

Permutation is used in a situation where order of arrangement matters.
This also means repetition of same sequence in different order is allowed.

\subsection{Combinations}
Combinations is written in the format:
\begin{equation}
	\Comb{n}{r} = \frac{n!}{r!(n-r)!}
\end{equation}
Where $n$ is the objects to chose from and $r$ is the number of objects chosen.

Combination is used in a situation where order does not matter.
This also means that repetitions of the same sequence is not allowed.

\subsection{Binomial Theorem}
Binomial Theorem models the resulted equation from $(a+b)^n$.
\begin{equation}
	(a+b)^n = \Comb{n}{0} a^n b^0 + \Comb{n}{1} a^{n-1} b^1 + \Comb{n}{2} a^{n-2} b^2 + \cdots + \Comb{n}{r} a^{n-r} b^r + \cdots + \Comb{n}{n} a^0 x^n 
\end{equation}
Where $r$ is the r-th term inside the sequence.

Note, the prefix for each term (Combination value) also corresponds to Pascal's Triangle.

\begin{figure}[h]
\centering
	\begin{tabular}{>{$r=}l<{$\hspace{12pt}}*{13}{c}}
	0 &&&&&&&1&&&&&&\\
	1 &&&&&&1&&1&&&&&\\
	2 &&&&&1&&2&&1&&&&\\
	3 &&&&1&&3&&3&&1&&&\\
	4 &&&1&&4&&6&&4&&1&&\\
	5 &&1&&5&&10&&10&&5&&1&\\
	6 &1&&6&&15&&20&&15&&6&&1
	\end{tabular}
	\caption{Pascal's Triangle}
\end{figure}

We can find the specific term from binomial expansion by using this expression:
\begin{equation}
	\Comb{n}{r} a^r b^{n-r}
\end{equation}

A more generalized Binomial Theorem that supports negative exponents can be found here:
\begin{equation}
	(1+x)^a = 1 + ax + \frac{a(a-1)}{2!} + \frac{a(a-1)(a-2)}{3!} + \cdots + \frac{a(a-1)(a-2)\cdots(a-r+1)}{r!} x^r + \cdots
\end{equation}
We can use this expression to manually calculate roots of a number.
\begin{align*}
	\sqrt{3} = (1 + 2)^{\frac{1}{2}}
\end{align*} 

\section{Financial Applications}
Geometric Sequences is sometimes used to calculate the interest of a deposit or similar circumstances.
A equation to model this is:

\begin{equation}
	FV = PV \cdot ( 1 + \frac{r}{100k} )^{kn}
\end{equation}

Where $FV$ is the future value, $PV$ is the present value, $n$ is the number of years, $k$ is the compound periods per year, and  $r$ is the nominal annual rate of interest in percent.
If $r$ is negative, then there is exponential decay.

\end{document}
