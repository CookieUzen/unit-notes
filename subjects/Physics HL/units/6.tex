\documentclass[../notes.tex]{subfiles}

\begin{document}

\section{Circular Motion}
When an object is moving in a circle, it is undergoing circular motion.
A uniformed circular motion is when the linear velocity stays constant.
We will only be looking at uniformed circular motion in IB.

In circular motion, the linear velocity is constant, but velocity is always changing.
There is a constant acceleration towards the center of the circle.
The velocity of the object is always a tangent to the circle.

In order for a circular motion to exist, an object must be moving tangentially to another object, while a centripetal force is exerted on the object.

\subsection{Angular Displacement}
Angular displacement ($\theta$) is the angle an object travels in circular motion.
This is measured in Radians (rad).
To convert between radians and degrees:
\begin{equation}
	\pi\; rad = 180^\circ
\end{equation}

\subsection{Angular Velocity}
Angular velocity ($\omega$) measures the change in angle overtime.
\begin{equation}
	\omega = \frac{\theta}{t}
\end{equation}
Where $t$ is time.

\subsection{Period and Frequency}
Period ($T$) is defined as the time taken to complete one revolution around the circle.
The SI Unit of period is seconds ($s$).
\begin{equation}
	T = \frac{2\pi}{\omega}
\end{equation}
$2\pi$ is the total angle of the circle, and $\omega$ is the angular velocity.

Frequency ($f$) is defined as the revolutions that can be completed in one second.
The SI Unit of frequency is Hertz ($Hz$).
One hertz is one revolution per second.
\begin{equation}
	f = \frac{1}{T}
\end{equation}

\subsection{Linear and Angular Velocity}
Linear velocity measures the speed of rotation.
\begin{equation}
	v = \frac{2 \pi r}{T}
\end{equation}
Where $r$ as the radius of the circle.

The relationship between linear velocity and angular velocity is:
\begin{equation}
	v = \omega r
\end{equation}

\subsection{Centripetal Acceleration}
Centripetal acceleration is the acceleration of the object as it undergoes circular motion.
\begin{equation}
	a = \frac{v^2}{r} = \omega^2 r = v\omega
\end{equation}

\subsection{Centripetal Force}
Centripetal force is the resultant force required to keep an object in circular motion.
Similar to the acceleration, centripetal force points to the center of the circle.
\begin{equation}
	F = ma = m \frac{v^2}{r} = m \omega^2 r = mv\omega
\end{equation}

The centripetal force itself is not a force. 
It is a combination of other forces.
For example, when a car undergoes into circular motion, friction between the car and ground is the centripetal force.
For a satellite that is undergoing circular motion, its centripetal force is gravity. 

\section{Newton's Law of Gravitation}
Gravity is a force one mass exerts on another.

\subsection{Gravitational Field Strength}
Gravitational field strength is the force per unit mass experienced by a point mass placed at a specific point.
\begin{equation}
	g = \frac{F}{m}
\end{equation}
Where $g$ is the acceleration due to gravity, or gravitational field strength.
$F$ is the force due to gravity, and $m$ is the mass of the object.

\subsection{Newton's Law of Gravitation}
The force due to gravity can be calculated using the equation below.
\begin{equation}
	F = \frac{GMm}{r^2}
\end{equation}
Where $F$ is the force, $G$ is Universal Gravitational Constant ($6.67 \cdot 10^{-11}$), $m$ and $M$ are the mass of the objects, and $r$ is the distance from the center of mass.

If we plug this equation into the equation for gravitational field strength, we find another equation for gravitational field strength:
\begin{equation}
	g = \frac{GM}{r^2}
\end{equation}

\subsection{Orbit}
For an object to remain in orbit, it means it is circular motion with a planet.
Likely, the only centripetal force acting on the object is gravity due to the planet.
Thus, we can model this relationship in an equation.
\begin{equation}
	\frac{mv^2}{r} = \frac{GMm}{r^2}
\end{equation}

The orbit velocity of a satellite can be calculated using the equation above.
By simplifying, we get the resultant equation:
\begin{equation}
	V_{orbit} = \sqrt{\frac{GM}{r}}
\end{equation}


Further expansion of this equation can give us Kepler's third law.
\begin{equation}
	T^2 \propto r^3
\end{equation}

\end{document}
