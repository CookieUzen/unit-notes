\documentclass[../notes.tex]{subfiles}

\begin{document}

\section{Oscillations}
An oscillation is an action that is repeated.
If an oscillation is \texttt{isochronous}, it means the oscillation repeat in the same amount of time period.
In this periodic motion, the duration of each repetition (or \texttt{cycle}) is called the \texttt{period} ($T$, in seconds).
\texttt{Frequency} ($f$, in Hertz or $Hz$) is the inverse of period, measuring how many cycles can occur in one second.
\begin{equation}
    T = \frac{1}{f}
\end{equation}

\section{Traveling Waves}
Waves are a type of oscillation that caries energy and information.
A traveling wave can carry energy and information across space. 
In a traveling wave, particles move, hitting other particles. 
This forms a chain reaction, where each particle is connected with the next, moving the wave.

\subsection{Mechanical and Electromagnetic}
There are two fundamental types of waves, \texttt{mechanical waves} and \texttt{electromagnetic} waves.
Mechanical waves require a medium travel through, while electromagnetic waves do not.

Some common mechanical waves are: 
\begin{itemize}
    \item sound waves
    \item water waves
    \item wind waves
    \item vibrations
\end{itemize}

Some common electromagnetic waves belong on a spectrum:
\begin{tabular}{ l | c }
    Wave & Wavelength \\
    \hline
    Radio & $10^3 m$ \\
    Microwaves & $10^{-2} m$ \\ 
    Infrared & $10^{-5}$ \\
    Visible & $10^{-6}$ \\
    Ultraviolet & $10^{-8}$ \\
    X-ray & $10^{-10}$ \\
    Gamma ray & $10^{-12}$
\end{tabular}

The visible light spectrum starts at $400nm$ and ends at $700nm$.
Red is the color with the longest wavelength, and violet is the color with the shortest wavelength.

\subsection{Transverse and Longitudinal}
Waves are also categories into \texttt{longitudinal} and \texttt{transverse} waves.
In a longitudinal wave, the particles of the wave travel \textit{horizontal} to the direction of the wave.
In a transverse wave, the particles of the wave travel \textit{perpendicular} to the direction of the wave.

Light is a transverse wave, while sound is a longitudinal wave.

\subsection{Describing Waves}
\begin{itemize}
    \item \texttt{Wavelength} $\lambda$ is the shortest distance between two points that are in phase on a wave.
    \item \texttt{Amplitude} $A$ is the maximum displacement of a wave from its resting/equilibrium position.
    \item \texttt{Resting/Equilbrium Postion} is where the wave is not displaced.
    \item Crest
    \item Trough
    \item Phase
\end{itemize}

\subsection{Graphs}

\subsubsection{Longitudinal}

\subsubsection{Transverse}

\subsubsection{Displacement Time}

\section{Wave Characteristics}

\end{document}
