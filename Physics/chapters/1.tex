\documentclass[../notes.tex]{subfiles}

\begin{document}

\section{Measurements in Physics}
\subsection{Physical Quantities and Units}

\subsubsection{Physical Quantities}
All measurements are made of the \textit{seven} basic quantities.
The table shows their quantities and their units.

\begin{center}
	\begin{tabular}{c c c}
		Quantity            & SI Unit  & Symbol \\
		\hline
		Time                & second   & $s$ \\
		Distance            & meter    & $m$ \\
		Mass                & kilogram & $kg$ \\
		Current             & ampere   & $A$ \\
		Temperature         & kelvin   & $K$ \\
		Amount of substance & mole     & $mol$ \\
		Luminous intensity  & candela  & $cd$ \\
	\end{tabular}
\end{center}

\subsubsection{Derived Quantities}
These quantities are derived from and \texttt{Physical Quantities}.
The table below show some common derived quantities.

\begin{center}
	\begin{tabular}{c c c c c}
		Quantity     & Symbol & Unit                        & SI Unit & Usual Symbol \\
		\hline
		Velocity     & $v$    & $m \cdot s^{-1}$                &               & \\
		Acceleration & $a$    & $m \cdot s^{-2}$                &               & \\
		Force        & $F$    & $kg \cdot m \cdot s^{-2}$       & Newton ($N$)  & \\
		Energy, Work & $E, W$ & $kg \cdot m^2 \cdot s^{-2}$     & Joule ($J$)         &  \\
		Power        & $P$    & $kg \cdot m^2 \cdot s^{-3}$     & Watt ($W$)    & $J \cdot s^{-1}$ \\
		Momentum     & $p$    & $kg \cdot m^{-1} \cdot s^{-2}$  &               & \\
		Pressure     & $p$    & $kg \cdot m^{-1} \cdot s^{-2} $ & Pascal ($Pa$) & $N \cdot m^{-2}$ \\
		Gravitation Field Strength & $g$ & $m \cdot s^{-2}$ & & $N \cdot kg^{-1}$ \\
		Electric Resistance & $R$ & $kg \cdot m^2 \cdot s^{-3} \cdot A^{-2}$ & Ohm ($\Omega$) & $V \cdot A^{-1}$ \\
		Frequency & $f$ & $s^{-1}$ & herts ($Hz$) & \\
	\end{tabular}
\end{center}

\subsection{Physical Quantities and Calculations}
When plugging in quantities into an equation, always be aware of unit conversion.
A method to avoid calculation error with unit conversion is to treat the units as mathematical entities.
This is called quantity calculus.

\subsection{Scientific Notation}
Scientific Notation leaves only one digit, and concatenate the rest as powers of ten.
For example: $4.82 \cdot 10^4$.
The decimal places indicates how many digits of significant figures is the number accurate to.

\subsubsection{SI Prefix}
A common list of SI Prefixes can be found below:
\begin{center}
     	 \begin{tabular}{c c c c c c}
			Factor  & Name  & Symbol & Factor     & Name  & Symbol \\
			\hline
			$10^1$  & deca  & $da$   & $10^{-1}$  & deci  & $d$ \\
			$10^2$  & hecto & $h$    & $10^{-2}$  & centi & $c$ \\
			$10^3$  & kilo  & $k$    & $10^{-3}$  & milli & $m$ \\
			$10^6$  & mega  & $M$    & $10^{-6}$  & micro & $\mu$ \\
			$10^9$  & giga  & $G$    & $10^{-9}$  & nano  & $n$ \\
			$10^12$ & tera  & $T$    & $10^{-12}$ & pico  & $p$ \\
			$10^15$ & peta  & $P$    & $10^{-15}$ & femto & $f$ \\
			$10^18$ & exa   & $E$    & $10^{-18}$ & atto  & $a$ \\
			$10^21$ & zetta & $Z$    & $10^{-21}$ & zepto & $z$ \\
			$10^24$ & yotta & $Y$    & $10^{-24}$ & yocto & $y$ \\
     	 \end{tabular}
\end{center}

\subsubsection{Significant Figures}
Significant Figures determines the amount of digits that is accurate in a certain calculation.
To find the significant figure of a number, we use the following rules:
\begin{itemize}
	\item Left most \textit{non-zero} digit is the \texttt{most significant digit}
	\item If there is no decimal points, the rightmost digit \textit{non zero} digit is the \texttt{least significant digit}
	\item If there is a decimal point, the rightmost digit is the \texttt{least significant digit}
	\item All digits between the \texttt{most significant digit} and the \texttt{least significant digit} is significant.
\end{itemize}

When doing a calculation, always round to the smallest significant figure.
When rounding, round up with numbers above and including 5, and down with numbers below 5.

\subsection{Uncertainties and Error}
In physics, there will always be issues with uncertainty and errors.
It is important to take these into account when doing calculations with the data.

\subsection{Quantifying Uncertainties}

\subsection{Combining Uncertainties}

\subsection{Displaying Uncertainties}

\section{Vectors and Scalers}
Physical quantities can be grouped into two different categories, scalers and vectors.

\subsection{Define Scalers and Vectors}
\textbf{Scalers} are unit of measurement that only describe the magnitude of an object.
For example, distance is a scaler: it only describe how far you went.

\textbf{Vectors} are a unit of measurement that describe both the magnitude and a direction.
For example, displacement is a vector. 
It describe the direction of movement and the distance traveled.

Below is a list of common scalers and vectors:

\begin{center}
	\begin{tabular}{c|c}
		Scaler & Vector \\ 
		\hline
		speed & velocity \\
		distance & displacement \\
		energy, work, power & acceleration \\
		temperature & force \\
		pressure & momentum \\
		mass & impulse \\
		volume & electric field strength \\
	\end{tabular}
\end{center}

\subsection{Working with Vectors}
We tend to break vectors into the x direction and the y direction.
This way, we can represent the vectors with positive and negative values.
For example, a force applied on a object upwards can be indicated with a positive sign, and a downwards force represented by a negative sign.

To separate the X and Y direction, we usually use trig function (sin cos tan).

\end{document}
