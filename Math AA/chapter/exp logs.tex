\documentclass[../notes.tex]{subfiles}

\begin{document}

\section{Exponents}

\subsection{Identities}
Exponents have some simply identities.

\begin{align}
	\begin{dcases}
		a^m \cdot a^n = a^{m+n} \\
		a^m \div a^n = a^{m-n} \\
	\end{dcases} \\
	\begin{dcases}
		a^m \cdot b^n = (ab)^m \\
		a^m \div a^m = (\frac{a}{b})^m \\
	\end{dcases} \\
	(a^m)^n &= a^{mn} \\
	a^0 &= 1 \\
	a^{\frac{1}{2}} &= \sqrt{a} \\
	a^{-m} &= \frac{1}{a^m} \\
	a^{\frac{m}{n}} &= \sqrt[n]{m}
\end{align} 

\subsection{Solving Exponents}
When solving exponents, always look for similar base or similar exponents.

\section{Logs}
\begin{equation}
	\log_a{b}
\end{equation}
$b$ has to be greater than or equal to 0.

\subsection{Identities}
\begin{align}
	\begin{dcases}
	\log_a{b} = x \\
	a^x = b
	\end{dcases} \\
	\log_a{1} &= 0 \\
	a^{\log_a{x}} &= x \\
	\log_a{x} + \log_a{y} &= log_a{x\cdot y} \\
	\log_a{x} - \log_a{y} &= log_a{x\div y} \\
	\log_a{x^b} &= b\cdot\log_a{x} \\
	\log_{a^b}{x} &= \frac{1}{b}\cdot\log_a{x} \\
	\log_m{n} &= \frac{\log_x{n}}{\log_x{m}}
\end{align} 

\end{document}
